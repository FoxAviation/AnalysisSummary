\documentclass[paper=a4,paper=landscape, fontsize=9pt,DIV=25, twoside]{scrartcl}
%\usepackage[DIV=15]{typearea}


%Text libs
\usepackage{amsfonts}
%\usepackage[ansinew, utf8]{inputenc}
\usepackage{amsmath,amssymb,amsthm}
\usepackage{multicol}
\usepackage{enumitem}

\linespread{0.5}

%Math commands
\newcommand{\rplus}{{\mathbb{R}}^+}
\newcommand{\real}{{\mathbb{R}}}
\newcommand{\erw}{{\mathbb{E}}}
\newcommand{\nat}{\mathbb{N}}
\newcommand{\natnull}{\mathbb{N}_0}
\newcommand{\laO}{\mathcal{O}}

% Redefine section commands to use less space
\makeatletter
\renewcommand{\section}{\@startsection{section}{1}{0mm}%
                                {-1ex plus -.5ex minus -.2ex}%
                                {0.5ex plus .2ex}%x
                                {\normalfont\large\bfseries}}
\renewcommand{\subsection}{\@startsection{subsection}{2}{0mm}%
                                {-1explus -.5ex minus -.2ex}%
                                {0.5ex plus .2ex}%
                                {\normalfont\normalsize\bfseries}}
\renewcommand{\subsubsection}{\@startsection{subsubsection}{3}{0mm}%
                                {-1ex plus -.5ex minus -.2ex}%
                                {1ex plus .2ex}%
                                {\normalfont\small\bfseries}}
\makeatother

% Don't print section numbers
\setcounter{secnumdepth}{0}


\setlength{\parindent}{0pt}
\setlength{\parskip}{0pt plus 0.0ex}

\begin{document}
\begin{multicols*}{4}


	\section{Folgen}
		\paragraph{Def: 2.2 - Grenzwert einer reellen Folge}
			\begin{itemize}[noitemsep,nolistsep]
				\item $a \in \real$ Grenzwert von $(a_n)$ $\Leftrightarrow \forall \epsilon > 0$ $\exists n_0 \in \nat$ $\forall n \geq n_0:$ $|a_n - a| < \epsilon$
				\item Existiert $a \in \real$ Grenzwert $\Rightarrow$ $(a_n)$ konvergent, sonst $(a_n)$ divergent
			\end{itemize}
		\paragraph{Satz 2.3 - Rechenregeln für Grenzwerte}
			$(a_n)_{n \in \nat}$, $(b_n)_{n \in \nat}$ reelle Folgen, $\lim\limits_{n \rightarrow \infty} a_n = a$, $\lim\limits_{n \rightarrow \infty} b_n = b$
			\begin{itemize}[noitemsep,nolistsep]
				\item Folge $(a_n + b_n)$ konvergiert gegen $a+b$
				\item Folge $(a_n \cdot b_n)$ konvergiert gegen $a \cdot b$
				\item $b \neq 0 \Rightarrow (\frac{a_n}{b_n})_{n \in \nat}$ konvergiert gegen $\frac{a}{b}$ 
				\item $a_n \leq b_n$ für fast alle $n \in \nat \rightarrow a \leq b$
				\item \textbf{Einschließungskriterium} $a=b$, c reelle Folge und $a_n \leq c_n \leq b_n$ für fast alle $n \in \nat \Rightarrow$ $(c_n)_{n \in \nat}$ konvergiert gegen $a$
			\end{itemize}
		\textit{Spezialfall des Einschließungskriteriums:}\\$(x_n)_{n \in \nat} Folge, x \in R, (y_n)_{n \in \nat}$ Nullfolge, sodass $|x_n -x| \leq y_n$ für fast alle $n \Rightarrow (x_n)_{n \in \nat}$ konvergiert gegen $x$
		\paragraph{Satz 2.4 - Eigenschaften konvergenter Folgen}
		Sei $a_n$ konvergente reelle Folge
		\begin{itemize}[noitemsep,nolistsep]
			\item $(a_n)$ beschränkt
			\item $(a_n)$ besitzt genau einen Grenzwert
		\end{itemize}
		\paragraph{Def: 2.5 - Uneigentliche Konvergenz}
		$(a_n)_{a \in \nat}$ konvergiert uneigentlich gegen $\infty \Leftrightarrow \forall K > 0 \exists n_0 \in \nat \forall n \geq n_0: a_n > K$\\
		$(a_n)_{n \in \nat}$ konvergiert uneigentlich gegen $-\infty \Leftrightarrow (-a_n)_{n \in \nat}$ konvergiert uneigentlich gegen $\infty$
		\paragraph{Satz 2.6 - Rechenregeln für uneigentliche Konvergenz}
		$(b_n)_{n \in \nat}$ reelle Folge, $\lim\limits_{n \rightarrow \infty} (b_n)_{n \in \nat} = \infty$, $(a_n)_{n \in \nat}$ reelle Folge, $\lim\limits_{n \rightarrow \infty} a_n = a, a \in \real \cup \{\infty, -\infty\}$
		\begin{itemize}[noitemsep,nolistsep]
			\item $a \neq - \infty \Rightarrow (a_n + b_n)_{n \in \nat}$ konvergiert uneigentlich gegen $\infty$
			\item $a \neq 0 \Rightarrow (a_n + b_n)_{n \in \nat}$ konvergiert uneigentlich
			\item $a > 0 \Rightarrow \lim\limits_{n \rightarrow \infty} a_nb_n = \infty$
			\item $a < 0 \Rightarrow \lim\limits_{n \rightarrow \infty} a_nb_n = -\infty$
			\item $a \notin \{\infty, -\infty\} \Rightarrow (\frac{a_n}{b_n})_{n \in \nat}$ konvergiert gegen 0
		\end{itemize}
		\paragraph{Def 2.7 - Monotone Folgen}
		$(a_n)_{n \in \nat}$ reelle Folge heißt
		\begin{itemize}[noitemsep,nolistsep]
			\item monoton wachsend, falls $a_{n+1} \geq a_n \forall n \in \nat$
			\item streng monoton wachsend, falls  $a_{n+1} > a_n \forall n \in \nat$
			\item monoton fallend, falls  $a_{n+1} \leq a_n \forall n \in \nat$
			\item streng monoton fallend, falls  $a_{n+1} < a_n \forall n \in \nat$
		\end{itemize}
		\paragraph{Satz 2.8 - Monotoniesatz}
		$(a_n)_{n \in \nat}$ reelle Folge, wachsend und nach oben beschränkt $\Rightarrow$ $(a_n)_{n \in \nat}$ konvergent und $\lim\limits_{n \rightarrow \infty} a_n = sup_{n \in \nat} a_n := sup \{a_n : n \in \nat\}$
		\paragraph{Def 2.9 Häufungspunkt} a $\in \real$ Häufungspunkt $\Leftrightarrow$ $ \exists (a_{n_k})_{k \in \nat}$ Teilfolge von $(a_n)_{n \in \nat}$, die gegen a konvergiert.
		\paragraph{Satz von Bolzano-Weierstraß} Jede beschränkte reelle Folge $(an)_{n \in \nat}$ besitzt eine konvergente Teilfolge und hat also mindestens
		einen Häufungspunkt.
		\paragraph{Def 2.11 - Limes superior, limes inferior}
		$(a_n)_{n \in \nat}$ nach oben (unten) beschänkt $\Rightarrow$ größter (kleinster) Häufungspunkt: \textbf{Limes superior (inferior)}
	\section{Komplexe und mehrdimensionale Folgen}
		\paragraph{Def 3.1 Grenzwert komplexer Folgen}
		z GW von $(z_n) \Leftrightarrow \forall \epsilon > 0 \exists n_0 \in \nat \forall n \geq n_0 : |z_n - z| < \epsilon$\\
		$\exists GW \Leftarrow z_n$ konvergent
		\paragraph{Konvergenz} $(z_n)_{n \in \nat} = a_n + ib_n: \lim\limits_{n\rightarrow \infty} z_n = \lim\limits_{n\rightarrow \infty} a_n + i \lim\limits_{n\rightarrow \infty} b_n$
		\paragraph{Grenzwert}
		$\lim\limits_{n\rightarrow \infty} v_n = v \Leftrightarrow \lim\limits_{n\rightarrow \infty} ||v_n - v||_2 = 0$
                \section{Reihen}

              \end{multicols*}
\end{document}
%%% Local Variables:
%%% mode: latex
%%% TeX-master: t
%%% End:
