\documentclass[paper=a4,paper=landscape, fontsize=6pt,DIV=25, twoside]{scrartcl}
%\usepackage[DIV=15]{typearea}


%Text libs
\usepackage{fontspec}
\usepackage{amsfonts}
\usepackage{amsmath,amssymb,amsthm}
\usepackage{multicol}
\usepackage{enumitem}
\usepackage{color}

\linespread{0.5}

%Math commands
\newcommand{\rplus}{{\mathbb{R}}^+}
\newcommand{\real}{{\mathbb{R}}}
\newcommand{\compl}{\mathbb{C}}
\newcommand{\erw}{{\mathbb{E}}}
\newcommand{\nat}{\mathbb{N}}
\newcommand{\natnull}{\mathbb{N}_0}
\newcommand{\laO}{\mathcal{O}}
\newcommand{\aseq}{(a_n)_{n \in \nat}}
\newcommand{\srow}{(s_n)_{n \in \nat}}

% Redefine section commands to use less space
\makeatletter
\renewcommand{\section}{\@startsection{section}{1}{0mm}%
                                {-1ex plus -.5ex minus -.2ex}%
                                {0.5ex plus .2ex}%x
                                {\normalfont\large\bfseries}}
\renewcommand{\subsection}{\@startsection{subsection}{2}{0mm}%
                                {-1explus -.5ex minus -.2ex}%
                                {0.5ex plus .2ex}%
                                {\normalfont\normalsize\bfseries}}
\renewcommand{\subsubsection}{\@startsection{subsubsection}{3}{0mm}%
                                {-1ex plus -.5ex minus -.2ex}%
                                {1ex plus .2ex}%
                                {\normalfont\small\bfseries}}
\makeatother

% Don't print section numbers
\setcounter{secnumdepth}{0}


\setlength{\parindent}{0pt}
\setlength{\parskip}{0pt plus 0.0ex}

\begin{document}
\begin{multicols*}{4}

%ADD sup, infimum, ....
	\section{Folgen}
		\paragraph{Def: 2.2 - Grenzwert einer reellen Folge}
			\begin{itemize}[noitemsep,nolistsep]
				\item $a \in \real$ Grenzwert von $(a_n)$ $\Leftrightarrow \forall \epsilon > 0$ $\exists n_0 \in \nat$ $\forall n \geq n_0:$ $|a_n - a| < \epsilon$
				\item Existiert $a \in \real$ Grenzwert $\Rightarrow$ $(a_n)$ konvergent, sonst $(a_n)$ divergent
			\end{itemize}
		\paragraph{Satz 2.3 - Rechenregeln für Grenzwerte}
			$(a_n)_{n \in \nat}$, $(b_n)_{n \in \nat}$ reelle Folgen, $\lim\limits_{n \rightarrow \infty} a_n = a$, $\lim\limits_{n \rightarrow \infty} b_n = b$
			\begin{itemize}[noitemsep,nolistsep]
				\item Folge $(a_n + b_n)$ konvergiert gegen $a+b$
				\item Folge $(a_n \cdot b_n)$ konvergiert gegen $a \cdot b$
				\item $b \neq 0 \Rightarrow (\frac{a_n}{b_n})_{n \in \nat}$ konvergiert gegen $\frac{a}{b}$ 
				\item $a_n \leq b_n$ für fast alle $n \in \nat \rightarrow a \leq b$
				\item \textbf{Einschließungskriterium} $a=b$, c reelle Folge und $a_n \leq c_n \leq b_n$ für fast alle $n \in \nat \Rightarrow$ $(c_n)_{n \in \nat}$ konvergiert gegen $a$
			\end{itemize}
		\textit{Spezialfall des Einschließungskriteriums:}\\$(x_n)_{n \in \nat} Folge, x \in R, (y_n)_{n \in \nat}$ Nullfolge, sodass $|x_n -x| \leq y_n$ für fast alle $n \Rightarrow (x_n)_{n \in \nat}$ konvergiert gegen $x$
		\paragraph{Satz 2.4 - Eigenschaften konvergenter Folgen}
		Sei $a_n$ konvergente reelle Folge
		\begin{itemize}[noitemsep,nolistsep]
			\item $(a_n)$ beschränkt
			\item $(a_n)$ besitzt genau einen Grenzwert
		\end{itemize}
		\paragraph{Def: 2.5 - Uneigentliche Konvergenz}
		$(a_n)_{a \in \nat}$ konvergiert uneigentlich gegen $\infty \Leftrightarrow \forall K > 0 \exists n_0 \in \nat \forall n \geq n_0: a_n > K$\\
		$\aseq$ konvergiert uneigentlich gegen $-\infty \Leftrightarrow (-a_n)_{n \in \nat}$ konvergiert uneigentlich gegen $\infty$
		\paragraph{Satz 2.6 - Rechenregeln für uneigentliche Konvergenz}
		$(b_n)_{n \in \nat}$ reelle Folge, $\lim\limits_{n \rightarrow \infty} (b_n)_{n \in \nat} = \infty$, $\aseq$ reelle Folge, $\lim\limits_{n \rightarrow \infty} a_n = a, a \in \real \cup \{\infty, -\infty\}$
		\begin{itemize}[noitemsep,nolistsep]
			\item $a \neq - \infty \Rightarrow (a_n + b_n)_{n \in \nat}$ konvergiert uneigentlich gegen $\infty$
			\item $a \neq 0 \Rightarrow (a_n + b_n)_{n \in \nat}$ konvergiert uneigentlich
			\item $a > 0 \Rightarrow \lim\limits_{n \rightarrow \infty} a_nb_n = \infty$
			\item $a < 0 \Rightarrow \lim\limits_{n \rightarrow \infty} a_nb_n = -\infty$
			\item $a \notin \{\infty, -\infty\} \Rightarrow (\frac{a_n}{b_n})_{n \in \nat}$ konvergiert gegen 0
		\end{itemize}
		\paragraph{Def 2.7 - Monotone Folgen}
		$\aseq$ reelle Folge heißt
		\begin{itemize}[noitemsep,nolistsep]
			\item monoton wachsend, falls $a_{n+1} \geq a_n \forall n \in \nat$
			\item streng monoton wachsend, falls  $a_{n+1} > a_n \forall n \in \nat$
			\item monoton fallend, falls  $a_{n+1} \leq a_n \forall n \in \nat$
			\item streng monoton fallend, falls  $a_{n+1} < a_n \forall n \in \nat$
		\end{itemize}
		\paragraph{Satz 2.8 - Monotoniesatz}
		$\aseq$ reelle Folge, wachsend und nach oben beschränkt $\Rightarrow$ $\aseq$ konvergent und $\lim\limits_{n \rightarrow \infty} a_n = sup_{n \in \nat} a_n := sup \{a_n : n \in \nat\}$
		\paragraph{Def 2.9 Häufungspunkt} a $\in \real$ Häufungspunkt $\Leftrightarrow$ $ \exists (a_{n_k})_{k \in \nat}$ Teilfolge von $\aseq$, die gegen a konvergiert.
		\paragraph{Satz von Bolzano-Weierstraß} Jede beschränkte reelle Folge $(an)_{n \in \nat}$ besitzt eine konvergente Teilfolge und hat also mindestens
		einen Häufungspunkt.
		\paragraph{Def 2.11 - Limes superior, limes inferior}
		$\aseq$ nach oben (unten) beschänkt $\Rightarrow$ größter (kleinster) Häufungspunkt: \textbf{Limes superior (inferior)}
	\section{Komplexe und mehrdimensionale Folgen}
		\paragraph{Def 3.1 Grenzwert komplexer Folgen}
		z GW von $(z_n) \Leftrightarrow \forall \epsilon > 0 \exists n_0 \in \nat \forall n \geq n_0 : |z_n - z| < \epsilon$\\
		$\exists GW \Leftarrow z_n$ konvergent
		\paragraph{Konvergenz} $(z_n)_{n \in \nat} = a_n + ib_n: \lim\limits_{n\rightarrow \infty} z_n = \lim\limits_{n\rightarrow \infty} a_n + i \lim\limits_{n\rightarrow \infty} b_n$
		\paragraph{Grenzwert}
		$\lim\limits_{n\rightarrow \infty} v_n = v \Leftrightarrow \lim\limits_{n\rightarrow \infty} ||v_n - v||_2 = 0$
	\section{Reihen}
		\paragraph{Def 2.2 - Folgen Grenzwert in $\real$}
		$a$ \text{ Grenzwert von } $ (a_n) \Leftrightarrow \forall \epsilon > 0 \exists n_0 \in \nat \forall b \geq n_0 : |a_n - a| < \epsilon$
		\paragraph{Nullfolge}
		$\lim\limits_{n \rightarrow \infty} (a_n) \rightarrow 0$
		\paragraph{Rechenregeln Grenzwerte}
			\begin{itemize}
				\item $(a_n+b_n) \rightarrow a+b$
				\item $(a_n \cdot b_n) \rightarrow a \cdot b$
				\item $b \neq 0 \Rightarrow (\frac{a_n}{b_n}) \rightarrow \frac{a}{b}$
				\item \textbf{Einschließungskriterium:} $a=b \wedge a_n \leq c_n \leq b_n$ für fast alle $n \in \nat \Rightarrow c_n \rightarrow a$
			\end{itemize}
		\paragraph{Eigenschaften konvergenter Folgen}
		$(a_n)$ beschränkt $\Rightarrow \{a_n : n \in \nat\} beschränkt \wedge \exists! \text{ ein GW}$
		\paragraph{Uneigentliche Konvergenz}
		$\aseq$ divergent $\Rightarrow\aseq$ konvergiert uneig. gg. $\infty \Leftrightarrow \forall K > 0 \exists n_0 \in \nat \forall n \geq n_0: a_n > K$
		\paragraph{Rechenregeln uneig. Konvergenz}
			\begin{itemize}
				\item $a \neq -\inf \Rightarrow (a_n+b_n)_{n \in \nat} \rightarrow \infty$
				\item $a \neq 0 \Rightarrow (a_n \cdot b_n)_{n \in \nat}$ konvergiert
				\item $a \notin \{-\infty, \infty\} \vee\aseq \text{ beschränkt } \Rightarrow (\frac{a_n}{b_n})_{n \in \nat} \rightarrow 0$
			\end{itemize}
		\paragraph{Monotone Folge}
			$(a_n)$ monoton wachsend, falls $a_{n+1} \geq a_n \forall n \in \nat$. (Äquivalent für $>,<,\leq$)
		\paragraph{Monotoniesatz}
			$\aseq$ monoton wachsend $\wedge$ nach oben beschränkt $\Rightarrow \lim\limits_{n \rightarrow \infty} a_n = \underset{n \in \nat}{sup}\:a_n = sup\:\{a_n:n \in \nat\}$
		\paragraph{Teilfolge, Häufungspunkte}
			$(a_n)_{a \in \nat}$ reelle Folge:
			\begin{itemize}
			\item $(n_k)_{k \in \nat}$ streng monoton wachsend in $\nat \Rightarrow (a_{n_k})_{k \in \nat}$ Teilfolge von $\aseq$
			\item $a \in \real$ Häufungspunkt von $\aseq \Leftrightarrow \exists$ Teilfolge, die gg. $a$ konvergiert
			\end{itemize}
		\paragraph{Satz v. Bozano-Weierstraß}
		Jede beschränkte Folge $\aseq$ besitzt eine konvergente Teilfolge und hat min. einen Häufungspunkt
		\paragraph{Limes superior, Limes inferior}
		$(a_n)_{a \in \nat}$ reelle Folge: $\aseq$ nach oben (unten) beschränkt $\rightarrow$ Bez. größter (kleinster) Häufungspunkt: Limes superior (inferior)
	\section{Komplexe und mehrdimensionale Folgen}
		\paragraph{Grenzwert komplexer Folgen}
		$z$ GW von $(z_n) \Leftrightarrow \forall \epsilon > 0 \exists n_0 \in \nat \forall n \geq n_0: |z_n - z| < \epsilon$. Existiert $z \Leftarrow (z_n)$ konvergent.
		\paragraph{Konvergenz komplexer Folgen}
		\begin{itemize}
			\item $z_n = a_n + ib_n$ konvergiert $\Leftrightarrow$ $a_n$ und $b_n$ konvergieren
			\item $z_n$ konvergent $\Rightarrow \lim\limits_{n \rightarrow \infty} z_n = \lim\limits_{n \rightarrow \infty} a_n + i \cdot \lim\limits_{n \rightarrow \infty} b_n$
		\end{itemize}
		\paragraph{Grenzwert mehrdimensionaler Folgen}
		$\lim\limits_{n \rightarrow \infty} v_n = v \Leftrightarrow \lim\limits_{n \rightarrow \infty} \Arrowvert v_n - v\Arrowvert_2 = 0 \Leftrightarrow \lim\limits_{n \rightarrow \infty} \Arrowvert v_n - v\Arrowvert_\infty = 0$
	\section{Reihen}
		\paragraph{Konvergenz} $\srow$ konvergent gg. $s \in \compl \Leftrightarrow$ Folge der Partialsummen gg. $s$ konvergiert
		\paragraph{Majoranten- \& Minorantenkriterium}$\displaystyle b_n := \sum_{k=0}^{\infty} b_k; (b_k)_{k \in \nat}\:\text{relle Folge}; a_s := \sum_{k=0}^{\infty} a_k, |\aseq| \leq b_k$ für fast alle $k \in \nat$
		\begin{itemize}
			\item $b_s$ konvergiert $\Rightarrow a_s$ konvergiert absolut
			\item $a_s$ divergiert $\Rightarrow b_s$ divergiert
		\end{itemize}
		\paragraph{Quotientenkriterum}
		$\displaystyle \sum_{k=0}^{\infty} a_k, a \neq 0$ für fast alle $k \in \nat \wedge \lim\limits_{n \in \nat} \arrowvert \frac{a_{n+1}}{a_n} \arrowvert := q$ existiert $\Rightarrow$ (*)
		\paragraph{Wurzelkriterium}
		$\displaystyle \sum_{k=0}^{\infty} a_k;\:a_k \in \compl: q := \limsup\limits_{k \rightarrow \infty} \sqrt[k]{\arrowvert a_k \arrowvert} \Rightarrow$ (*)
		\paragraph{(*)}
		\begin{itemize}
			\item $q < 1 \Rightarrow$ Reihe konvergiert absolut
			\item $q > 1 \Rightarrow$ Reihe divergiert
		\end{itemize}
		\paragraph{Leibnitz-Kriterium}
		$\aseq$ relle, monoton fallende Nullfolge $\displaystyle \Rightarrow \sum_{k=0}^{\infty} (-1)^ka_k \Rightarrow \forall n \in \nat \lvert \sum_{k=0}^{\infty}(-1)^ka_k-s_n \rvert \leq  a_{n+1}$
		\paragraph{Umordnungssatz}\begin{itemize}
			\item Jede Umordnung einer konvergenten Reihe konvergiert gegen denselben Wert
			\item Konvergiert eine Reihe aus reellen Summanden, aber nicht absolut $\Rightarrow \forall s \in \real \exists$ bijektive Abbildung $\nat \rightarrow \nat$: die umgeordnete Reihe konvergiert gegen s
		\paragraph{Potenzreihe}
		$\displaystyle P(z) := \sum_{k=0}^{\infty} c_kz^k; c_k \in \compl; z \in \compl\\R := \frac{1}{\limsup\limits_{k \rightarrow \infty} \sqrt[k]{|c_k|}}$
		\end{itemize}
		\paragraph{Cauchy-Produkt}
		\color{red}{TODO: How to limit red to this expression????}
		\paragraph{Natürliche Exponentialfunktion}
		$\displaystyle exp(z) := \sum_{k=0}^{\infty} \frac{z^k}{k!}$
\end{multicols*}
\end{document}
%%% Local Variables:
%%% mode: latex
%%% TeX-master: t
%%% End:
