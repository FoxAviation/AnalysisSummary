\documentclass[paper=a3,paper=landscape, fontsize=9pt,DIV=25]{scrartcl}

% Text libs
\usepackage{fontspec}
\usepackage{amsfonts}
\usepackage{amsmath,amssymb,amsthm}
\usepackage{multicol}
\usepackage[ngerman]{babel}
\usepackage{enumitem}
\usepackage{color}
\usepackage{graphicx}
\usepackage{mathrsfs}
\usepackage{datetime}
\usepackage{fancyhdr}
\usepackage{color}
\usepackage{esint}

% header and footer
\pagestyle{fancy}
\fancyhf{}
\lhead{Analysis Cheatsheet WiSe 16/17}
\rfoot{\today\ \currenttime}
\lfoot{\thepage}

% line spacing
\linespread{0.5}



% Math commands
\newcommand{\rplus}{{\mathbb{R}}^+}
\newcommand{\real}{{\mathbb{R}}}
\newcommand{\compl}{\mathbb{C}}
\newcommand{\erw}{{\mathbb{E}}}
\newcommand{\nat}{\mathbb{N}}
\newcommand{\natnull}{\mathbb{N}_0}
\newcommand{\laO}{\mathcal{O}}
\newcommand{\aseq}{(a_n)_{n \in \nat}}
\newcommand{\srow}{(s_n)_{n \in \nat}}

% Redefine section commands to use less space
\makeatletter
\renewcommand{\section}{\@startsection{section}{1}{0mm}%
  {-1ex plus -.5ex minus -.2ex}%
  {0.5ex plus .2ex}%
  {\color{blue}\normalfont\large\bfseries}}
\renewcommand{\subsection}{\@startsection{subsection}{2}{0mm}%
  {-1explus -.5ex minus -.2ex}%
  {0.5ex plus .2ex}%
  {\normalfont\normalsize\bfseries}}



% Don't print section numbers
\setcounter{secnumdepth}{0}

% Paragraphs use less space
\setlength{\parindent}{0pt}
\setlength{\parskip}{0pt plus1pt minus1pt}

% make lists more compact
\setlist[itemize]{noitemsep,nolistsep, leftmargin=3mm}

% make superscript smaller
\DeclareMathSizes{9}{9}{5}{5}



\begin{document}
\begin{multicols*}{4}
	
	\section{Reelle Zahlen}
	$(K,+,\cdot)$ angeordneter Körper, $X \subset K$ nicht leer
	\paragraph{Sup, Inf, Max, Min}
	\begin{itemize}
		\item $s \in K$ \textbf{Sup(X)} $\Leftrightarrow$ s kleinste obere Schranke
		\item $s \in K$ \textbf{Inf(X)} $\Leftrightarrow$ s größte untere Schranke
		\item $s \in K$ \textbf{Max(x)} $\Leftrightarrow s =$ inf(X) $\wedge s \in K$
		\item $s \in K$ \textbf{Min(x)} $\Leftrightarrow s =$ inf(X) $\wedge s \in K$
	\end{itemize}
	\paragraph{$\epsilon-$Charakterisierung des Supremums}
	$s = \sup(X) \Leftrightarrow \forall \epsilon > 0 \exists x \in X: s-\epsilon < x \leq s$
	\textit{Entsprechendes gilt für das Infimum}
	\paragraph{Rechenregeln für sup}
	$X,Y \subset \real$ nach oben beschränkt:
	\begin{itemize}
		\item $\sup(X+Y) = \sup(X) + \sup(Y)$
		\item $\lambda > 0 \Rightarrow \sup(\lambda X) = \lambda \sup(X)$
		\item $X,Y \subset [0,\infty) \Rightarrow \sup(X) \cdot \sup(Y)$
		\item $X \subset Y \Rightarrow \sup(X) \leq \sup(Y)$
	\end{itemize}
\paragraph{Archimedische Anordnung}
\begin{itemize}
	\item $\real$ archimedisch angeordnet $\Rightarrow \forall x \in \real \exists n \in \nat: n > x$
	\item $\forall \epsilon > 0 \exists n \in \nat: \frac{1}{n} < \epsilon$
	\item $\forall a,b \in \real: a < b: \exists q \in \mathbb{Q}: a<q<b$
\end{itemize}

  \section{Folgen}
  \paragraph{Def: 2.2 - Grenzwert einer reellen Folge}
  \begin{itemize}
  \item $a \in \real$ Grenzwert von $(a_n)$ $\Leftrightarrow \forall \varepsilon > 0$ $\exists n_0 \in \nat$ $\forall n \geq n_0:$ $|a_n - a| < \varepsilon$
  \item Existiert $a \in \real$ Grenzwert $\Rightarrow$ $(a_n)$ konvergent, sonst $(a_n)$ divergent
  \end{itemize}

  \paragraph{Nullfolge}
$\lim\limits_{n \rightarrow \infty} (a_n) \rightarrow 0$

  \paragraph{Satz 2.3 - Rechenregeln für Grenzwerte}
  $(a_n)_{n \in \nat}$, $(b_n)_{n \in \nat}$ reelle Folgen, $\lim\limits_{n \rightarrow \infty} a_n = a$, $\lim\limits_{n \rightarrow \infty} b_n = b$
  \begin{itemize}
  \item Folge $(a_n + b_n)$ konvergiert gegen $a+b$
  \item Folge $(a_n \cdot b_n)$ konvergiert gegen $a \cdot b$
  \item $b \neq 0 \Rightarrow (\frac{a_n}{b_n})_{n \in \nat}$ konvergiert gegen $\frac{a}{b}$
  \item $a_n \leq b_n$ für fast alle $n \in \nat \rightarrow a \leq b$
  \item \textbf{Einschließungskriterium}\\$a=b \wedge a_n \leq c_n \leq b_n$ für fast alle $n \in \nat \Rightarrow c_n \rightarrow a$
  
  \end{itemize}

  \textit{Spezialfall des Einschließungskriteriums:}\\$(x_n)_{n \in \nat} Folge, x \in R, (y_n)_{n \in \nat}$ Nullfolge, sodass $|x_n -x| \leq y_n$ für fast alle $n \Rightarrow (x_n)_{n \in \nat}$ konvergiert gegen $x$


  \paragraph{Satz 2.4 - Eigenschaften konvergenter Folgen}\hspace{0pt} \\
  Sei $a_n$ konvergente reelle Folge
  \begin{itemize}
  \item $(a_n)$ beschränkt
  \item $(a_n)$ besitzt genau einen Grenzwert
  \end{itemize}


  \paragraph{Def: 2.5 - Uneigentliche Konvergenz}\hspace{0pt} \\
  $(a_n)_{a \in \nat}$ konvergiert uneigentlich gegen $\infty \Leftrightarrow \forall K > 0 \; \exists n_0 \in \nat \forall n \geq n_0: a_n > K$\\
  $\aseq$ konvergiert uneigentlich gegen $-\infty \Leftrightarrow (-a_n)_{n \in \nat}$ konvergiert uneigentlich gegen $\infty$

  \paragraph{Satz 2.6 - Rechenregeln für uneigentliche Konvergenz}\hspace{0pt} \\
  $(b_n)_{n \in \nat}$ reelle Folge, $\lim\limits_{n \rightarrow \infty} (b_n)_{n \in \nat} = \infty$, $\aseq$ reelle Folge, $\lim\limits_{n \rightarrow \infty} a_n = a, a \in \real \cup \{\infty, -\infty\}$

  \begin{itemize}
  \item $a \neq -\inf \Rightarrow (a_n+b_n)_{n \in \nat} \rightarrow \infty$
  \item $a \neq 0 \Rightarrow (a_n \cdot b_n)_{n \in \nat}$ konvergiert
  \item $a > 0 \Rightarrow \lim\limits_{n \rightarrow \infty} a_nb_n = \infty$
  \item $a < 0 \Rightarrow \lim\limits_{n \rightarrow \infty} a_nb_n = -\infty$
  \item $a \notin \{-\infty, \infty\} \vee\aseq \text{ beschränkt } \Rightarrow (\frac{a_n}{b_n})_{n \in \nat} \rightarrow 0$
  \end{itemize}

  \paragraph{Def 2.7 - Monotone Folgen}
  $\aseq$ reelle Folge heißt
  \begin{itemize}
  \item monoton wachsend, falls $a_{n+1} \geq a_n, \forall n \in \nat$
  \item streng monoton wachsend, falls  $a_{n+1} > a_n, \forall n \in \nat$
  \item monoton fallend, falls  $a_{n+1} \leq a_n, \forall n \in \nat$
  \item streng monoton fallend, falls  $a_{n+1} < a_n, \forall n \in \nat$
  \end{itemize}
 
 \paragraph{Monotoniesatz}
 $\aseq$ monoton wachsend $\wedge$ nach oben beschränkt $\Rightarrow \lim\limits_{n \rightarrow \infty} a_n = \underset{n \in \nat}{sup}\:a_n = sup\:\{a_n:n \in \nat\}$


  \paragraph{Def 2.9 Häufungspunkt}
  a $\in \real$ Häufungspunkt $\Leftrightarrow$ $ \exists (a_{n_k})_{k \in \nat}$ Teilfolge von $\aseq$, die gegen a konvergiert.


 \paragraph{Satz v. Bozano-Weierstraß}\hspace{0pt} \\
 Jede beschränkte Folge $\aseq$ besitzt eine konvergente Teilfolge und hat min. einen Häufungspunkt


  \paragraph{Def 2.11 - Limes superior, limes inferior}
  $\aseq$ nach oben (unten) beschränkt $\Rightarrow$ größter (kleinster) Häufungspunkt: \emph{Limes superior (inferior)}



\section{Komplexe und mehrdimensionale Folgen}

\paragraph{Grenzwert komplexer Folgen}\hspace{0pt} \\
$z$ GW von $(z_n) \Leftrightarrow \forall \varepsilon > 0 \exists n_0 \in \nat \forall n \geq n_0: |z_n - z| < \varepsilon$. Existiert $z \Leftarrow (z_n)$ konvergent.


\paragraph{Konvergenz komplexer Folgen}
\begin{itemize}
	\item $z_n = a_n + ib_n$ konvergiert $\Leftrightarrow$ $a_n$ und $b_n$ konvergieren
	\item $z_n$ konvergent $\Rightarrow \lim\limits_{n \rightarrow \infty} z_n = \lim\limits_{n \rightarrow \infty} a_n + i \cdot \lim\limits_{n \rightarrow \infty} b_n$
\end{itemize}


\paragraph{Grenzwert mehrdimensionaler Folgen}\hspace{0pt} \\
$\lim\limits_{n \rightarrow \infty} v_n = v \Leftrightarrow \lim\limits_{n \rightarrow \infty} \Arrowvert v_n - v\Arrowvert_2 = 0 \Leftrightarrow \lim\limits_{n \rightarrow \infty} \Arrowvert v_n - v\Arrowvert_\infty = 0 $

  \section{Reihen}

  \paragraph{Konvergenz}
  $\srow$ konvergent gg. $s \in \compl \Leftrightarrow$ Folge der Partialsummen gg. $s$ konvergiert

  \paragraph{Teilfolge, Häufungspunkte}
  $(a_n)_{a \in \nat}$ reelle Folge:
  \begin{itemize}
  	\item $(n_k)_{k \in \nat}$ streng monoton wachsend in $\nat \Rightarrow (a_{n_k})_{k \in \nat}$ Teilfolge von $\aseq$
  	\item $a \in \real$ Häufungspunkt von $\aseq \Leftrightarrow \exists$ Teilfolge, die gg. $a$ konvergiert
  \end{itemize}

  \paragraph{Majoranten- \& Minorantenkriterium}\hspace{0pt} \\
  $ b_n := \sum_{k=0}^{\infty} b_k; (b_k)_{k \in \nat}\:\text{relle Folge}\\ a_s := \sum_{k=0}^{\infty} a_k, |\aseq| \leq b_k \text{für fast alle}\ k \in \nat$
  \begin{itemize}
  \item $b_s \text{konvergiert}\ \Rightarrow a_s \text{konvergiert absolut}$
  \item $a_s \text{divergiert}\ \Rightarrow b_s$ divergiert
  \end{itemize}


  \paragraph{Quotientenkriterum}\hspace{0pt} \\
  $ \sum_{k=0}^{\infty} a_k, a \neq 0$ für fast alle $k \in \nat \wedge \lim\limits_{n \in \nat} \arrowvert \frac{a_{n+1}}{a_n} \arrowvert := q$ existiert $\Rightarrow$ (*)


  \paragraph{Wurzelkriterium}\hspace{0pt} \\
  $ \sum_{k=0}^{\infty} a_k;\:a_k \in \compl: q := \limsup\limits_{k \rightarrow \infty} \sqrt[k]{\arrowvert a_k \arrowvert} \Rightarrow$ (*)


  \paragraph{(*)}
  \begin{itemize}
  \item $q < 1 \Rightarrow$ Reihe konvergiert absolut
  \item $q > 1 \Rightarrow$ Reihe divergiert
  \end{itemize}


  \paragraph{Leibnitz-Kriterium}
  $\aseq$ relle, monoton fallende Nullfolge $ \Rightarrow \sum_{k=0}^{\infty} (-1)^ka_k \Rightarrow \forall n \in \nat \lvert \sum_{k=0}^{\infty}(-1)^ka_k-s_n \rvert \leq  a_{n+1}$


  \paragraph{Umordnungssatz}\begin{itemize}
  \item Jede Umordnung einer konvergenten Reihe konvergiert gegen denselben Wert
  \item Konvergiert eine Reihe aus reellen Summanden, aber nicht absolut $\Rightarrow \forall s \in \real \exists$ bijektive Abbildung $\nat \rightarrow \nat$: die umgeordnete Reihe konvergiert gegen s
\end{itemize}

  \paragraph{Potenzreihe}
    $ P(z) := \sum_{k=0}^{\infty} c_kz^k; c_k \in \compl; z \in \compl\\R := \frac{1}{\limsup\limits_{k \rightarrow \infty} \sqrt[k]{|c_k|}} :=$ Konvergenzradius

  \paragraph{Cauchy-Produkt}
  \begin{itemize}
  \item Seien $ \sum_{k=0}^{\infty} a_k,\:\sum_{k=0}^{\infty} b_k$ absolut konvergent $ \in \compl \Rightarrow (\sum_{k=0}^{\infty} a_k) (\sum_{k=0}^{\infty} b_k) = (\sum_{m=0}^{\infty} c_m)$ mit $ c_m=(\sum_{k=0}^{m} a_kb_{m-k})$ mit $c_k$ absolut konvergent.
  
  \item Seien $ \sum_{k=0}^{\infty} a_kz^k$, $\sum_{k=0}^{\infty} b_kz^k$ zwei Potenzreihen mit Konvergenzradien $R_a$ und $ R_b \Rightarrow (\sum_{k=0}^{\infty} a_kz^k)(\sum_{k=0}^{\infty} b_kz^k)=(\sum_{m=0}^{\infty} c_bz^m)$ mit $ c_m = \sum_{k=0}^{m} a_kb_{m-k}$ und Konvergenzradius $\min\{R_a,R_b\}$
  \end{itemize}


  \paragraph{Natürliche Exponentialfunktion}\hspace{0pt} \\
  $ exp(z) := \sum_{k=0}^{\infty} \frac{z^k}{k!} = \sum_{K=0}^{n}\binom{n}{k}\frac{1}{n^k}$


  \paragraph{Eigenschaften von exp}
  $\forall z,w \in \compl, x \in \real, n \in \nat$
  \begin{itemize}
  \item $\exp(z+w)=\exp(z)\cdot \exp(w)$
  \item $\exp(-z)=\frac{1}{\exp(z)}, \exp(z) \neq 0 \wedge \exp(\overline{z})=\exp(z)$
  \item $\lvert e^{ix} \rvert = 1$
  \item $\lim\limits_{n \rightarrow \infty} (1+\frac{z}{n})^n=\exp(z)$
  \item $ \lvert \exp(z)- \sum_{k=0}^{n} \frac{z^k}{k!} \rvert \leq 2 \cdot \frac{\lvert z \rvert ^{n+1}}{(n+1)!}$
  \item $ \lim\limits_{x \rightarrow 0} \frac{e^x-1}{x} = 1$
  \item $ \lim\limits_{x \rightarrow \infty} x^{-n}e^x=\infty$
  \item $ \lim\limits_{x \rightarrow -\infty} x^ne^x=0$
  \item $e^{i \frac{\pi}{2}} = i$
  \item $e^{i\pi}=-1$
  \item $e^{z+2\pi i}=e^z$
  \item $e^{ix}=\cos(x)+i\sin(x)$
  \end{itemize}

 %TODO: cosh, sinh, ...

  \paragraph{Trigonometrische Funktionen}\hspace{0pt} \\
  $\sin(z):=\frac{e^{iz}-e^{-iz}}{2i} = \sum_{k=0}^{\infty} (-1)^k \frac{z^{2k+1}}{(2k+1)!}\\
  \cos(z):=\frac{e^{iz}+e^{-iz}}{2} = \sum_{k=0}^{\infty} (-1)^k\frac{z^{2k}}{(2k)!}$


  \paragraph{Eigenschaften v. sin, cos, tan}
  $\forall z,w \in \compl, x \in \real$

  % TODO: Add more equations to the following list!!
  \begin{itemize}
  \item $\exp(iz) = \cos(z)+i\sin(w)$
  \item $(\sin(z))^2+(\cos(z))^2=1$
  \item $\sin(z+w)=\sin(z)\cos(w)+\cos(z)\sin(w)$
  \item $cos(z+w)=\cos(z)\sin(w)-\sin(z)\sin(w)$
  \item $\cos(x)=Re(e^{ix}) \; \sin(x)=Im(e^{ix})$
  \item $z=\tan(c): \arctan'(z)=\frac{1}{\tan'(c)}=\frac{1}{1+(\tan(c))^2}=\frac{1}{1+z^2}$
  \end{itemize}

  \resizebox{.9\hsize}{!}{
    \begin{tabular}{c|c|c|c|c|c}
      &$0$                     &$\frac{\pi}{6}=30°$               &$\frac{\pi}{4}=45°$                      &$\frac{\pi}{3}=60°$               &$\frac{\pi}{2}=90°$\\ \hline
      $\sin$ &$\frac{\sqrt{0}}{2}=0$  &$\frac{\sqrt{1}}{2}=\frac{1}{2}$  &$\frac{\sqrt{2}}{2}=\frac{1}{\sqrt{2}}$  &$\frac{\sqrt{3}}{2}$              &$\frac{\sqrt{4}}{2}=1$\\
      $\cos$ &$\frac{\sqrt{4}}{2}=1$  &$\frac{\sqrt{3}}{2}$              &$\frac{\sqrt{2}}{2}=\frac{1}{\sqrt{2}}$  &$\frac{\sqrt{1}}{2}=\frac{1}{2}$  &$\frac{\sqrt{0}}{2}=0$\\
    \end{tabular}
  }



  \section{Stetigkeit}

  \paragraph{Definition}\hspace{0pt} \\
  $\text{stetig in}\ c \Leftrightarrow \forall (x_n) \ \text{mit}\ \lim\limits_{x \rightarrow 0}x_n = c \ \text{gilt}\ \lim\limits_{x \rightarrow 0}f(x_n)=f(c)$

  \paragraph{Rechenregeln}
  $D \subseteq \real;f,g: D \rightarrow \real;f,g ~\text{stetig in}\ c \\ \Rightarrow f+g, f \cdot g,\frac{f}{g}$ ($g \neq 0$) stetig in $c$


  \paragraph{Komposition}
  $D,D' \subseteq \real, f:D \rightarrow \real$ stetig in $c$
  \begin{itemize}
  \item $y := f(c) \in D' \wedge g $ stetig in $y \Rightarrow (g \circ f): D \rightarrow \real$ stetig in $c$
  \item $f,g$ stetig $\wedge f(D) \subseteq D' \Rightarrow (g \circ f): D \rightarrow \real$ stetig
  \end{itemize}


  \paragraph{$\varepsilon$-$\delta$-Charakterisierung}
  $D \subseteq \real, f \rightarrow \real, c \in D \Rightarrow f$ stetig in $c \Leftrightarrow \forall \varepsilon > 0 \exists \delta > 0: \forall x \in D: \lvert x - c \rvert < \delta \Rightarrow \lvert f(x)-f(c)\rvert < \varepsilon$


  \paragraph{Zwischenwertsatz}
  $f:[a,b] \rightarrow \real$ stetig $\Rightarrow \forall y \in \real$ mit $ \min\{f(a),f(b)\} \leq y \leq \max\{f(a), f(b)\}: \exists x \in [a,b]: f(x)=y$


  \paragraph{Satz v. Max. und Min.}
  $[a,b]$ beschränkt, $f: [a,b] \rightarrow \real$ stetig
  \begin{itemize}
  \item $f$ beschränkt
  \item $\exists x_{\max}, x_{\min} \in [a,b]: f(x_{\max})=\sup\{f(x): x \in [a,b]\} \wedge f(x_{\min})=\inf\{f(x):x \in [a,b]\}$
  \end{itemize}

%TODO: Add Satz 5.8, lim von oben, lim von unten
  \paragraph{Stetigkeit in $\compl$ \& $\real ^n$} wörtlich übertragbar\\
  $D \subseteq \compl$ oder $D \subseteq \real^n$ abgeschlossen: $\forall f \text{ stetig}: D \rightarrow \compl$ bzw. $f: D \rightarrow \real^m$ beschränkt und nimmt auf D Maximum und Minimum an
  % \paragraph{Grenzwerte} ausgelassen

%TODO: Überarbeiten:
  \paragraph{Stetigkeit von Potenzreihen}
  $ f(z)=\sum_{k=0}^{\infty} c_kz^k$ mit Konvergenzradius $R \Rightarrow f: \{z: \lvert z \rvert < R \} \rightarrow \compl$ stetig


  \section{Differenziation}

  \paragraph{Definition}
  $I \subseteq \real, f: I \rightarrow \real$\\
$f'(c) := \lim\limits_{x \rightarrow c} \frac{f(x)-f(c)}{x-c}$

  \paragraph{Diff'barkeit $\Rightarrow$ Stetigkeit}
  $f$ diffbar in c $\Rightarrow f$ in $\compl$ stetig
  % TODO: Ableitungsregeln und Monotonie ausgelassen (wohl nicht nötig?)

  \paragraph{Monotonie \& Umkehrbarkeit}
  \begin{itemize}
  \item $f$ stetig: $f$ injektiv $\Leftrightarrow f$ streng monoton wachsend oder fallend
  \item $f$ außerdem surjektiv $\Rightarrow f^{-1}$ stetig $\wedge$ monoton wachsend / fallend
  \end{itemize}


  \paragraph{Differentiation v. $f^{-1}$}
  $f$ bijektiv, $f'(c) \neq 0 \Rightarrow z := f(c)$ diff'bar, $(f^{-1})'(z)=\frac{1}{f'(c)}$


  \paragraph{Logarithmus}
  $\lim\limits_{x\rightarrow0} \frac{\ln(1+x)}{x}=1$


%TODO: Definitionsbereich einfügen
  \paragraph{Diff. v. Potenzreihen}\hspace{0pt}\\
  $ f(x)=\sum_{k=0}^{\infty}c_kx^k \in \real: f'(x)=\sum_{k=1}^{\infty}kc_kx^{k-1}$, $D = \lvert R \rvert$


  \paragraph{Höhere Ableitungen}
  $\mathscr{C}^n(I)$: Vektorraum aller n-mal stetig diff'baren Funktionen $f: I \rightarrow \real$


  \paragraph{Extrema}
  \begin{itemize}
  \item \textbf{lokales Maximum} $\Leftrightarrow \varepsilon > 0: f(c) \geq f(x) \forall x \in (c-\varepsilon,c+\varepsilon)\cap I$
  \item \textbf{lokales Minimum} $\Leftrightarrow \varepsilon > 0: f(c) \leq f(x) \forall x \in (c-\varepsilon,c+\varepsilon)\cap I$
  \item \textbf{isoliertes lok. Max/Min} $\Leftrightarrow$ Max/min $\wedge x \neq c$
  \item \textbf{globales Max (Min)} $f(x) \geq (\leq) f(c)$
  \end{itemize}


  \paragraph{Satz von Rolle}
  $f: [a,b] \rightarrow \real, f(a)=f(b) \Rightarrow \exists \xi \in (a,b): f'(\xi)=0$


  \paragraph{Hinreichend für Extrema}
  $f$ diff'bar, $\exists c: f'(c)=0$
  \begin{itemize}
  \item f streng monoton wachsend um c $\Rightarrow$ f in c isol. lok. Min.
  \item $f \in \mathscr{C}^2, f''(c)=0 \Rightarrow f$ in c isol. lok. Min.
  \item f' um c str. monoton fallend $\Rightarrow f$ in c isol. lok. Max.
  \item $f \in \mathscr{C}^2, f''(c)<0 \Rightarrow f$ in c isol. lok. Max.
  \end{itemize}
  % L'Hopital ausgelassen%


  \paragraph{Taylor}\hspace{0pt} \\
  $ n \in \{\nat \cup \infty\}: T_nf(x;c):= \sum_{k=0}^{n}\frac{f^{(k)}(c)}{k!}(x-c)^k$
  \section{Integration}


  \paragraph{Riehmann-Integral}\hspace{0pt} \\
  $\varphi \in \tau [a,b]: \int_{a}^{b}\varphi dx := \sum_{k=1}^{n}c_k(x_k-x_{k-1})$


  \paragraph{Stetigkeit \& Monotonie $\Rightarrow integrierbar$} $f: [a,b] \rightarrow \real$
  \begin{itemize}
  \item $f$ stetig $\Rightarrow f$ integrierbar
  \item $f$ monoton $\Rightarrow f$ integrierbar
  \item $\exists$ Unterteilung v. $[a,b]$ stetig oder monoton $\Rightarrow f$ integrierbar
  \end{itemize}


  \paragraph{Mittelwertsatz d. Int.-R.}
  $f:[a,b] \rightarrow \real$ stetig, $\exists \xi \in [a,b]: \int_{a}^{b}f(x)dx=f(\xi)(b-a)$
  % Hauptsatz der Integralrechnung weggelassen (ev. weiter kürzen)


  \paragraph{Partielle Integration}
  $f,g: [a,b] \rightarrow \real$ stetig: $\int_{a}^{b}f(x)g'(x)dx=[f(x)g(x)]_a^b-\int_{a}^{b}f'(x)g(x) dx$


  \paragraph{Substitution}
  $\int_{a}^{b}f(g(x))g'(x)dx=\int_{g(a)}^{g(b)}f(y)dy$


  \paragraph{Majorantenkriterium f. Int.}\hspace{0pt} \\
  $f$ über $[a,b]$ absolut integrierbar $\Leftrightarrow$
  \begin{itemize}
  \item f in jedem Teilintervall $\in [a,b]$ integrierbar
  \item $\lvert f(x) \rvert \leq g(x) \forall x \in [a,b)$
  \item $g$ über $[a,b)$ uneig. integrierbar
  \end{itemize}

	\paragraph{Folgerung z. uneig. Integrierbarkeit}
	\begin{itemize}
		\item $f: (a,b] \rightarrow \real$ auf allen Teilintervallen $\in (a,b]$ integrierbar, $f(x)=\laO(\frac{1}{\lvert x-a \rvert ^s})$ für $x \rightarrow a$ mit $s \in [0,1) \Rightarrow$ f über $[a,b)$ uneig. integrierbar
		\item $f: [a,\infty) \rightarrow \real, a<b$ integrierbar, $f(x)=\laO(\frac{1}{x^s})$ mit $x \rightarrow \infty$, $s > 1 \Rightarrow f$ über $[a,\infty]$ uneig. integrierbar
	\end{itemize}


  \paragraph{Integralvergl.krit.}
  $f:[1,\infty]\rightarrow\real$ monoton fallend $, \forall x: f(x)>0, f$ über $[1,\infty]$ uneig. integr. $\Rightarrow \sum_{k=1}^{\infty}f(k)$ konvergiert
  % Ausgelassen: Satz von Taylor


  \paragraph{Potenzreihen}\hspace{0pt} \\
  $f(x)=\sum_{k=0}^{\infty}c_kx^k \Rightarrow \int f(x)dx=\sum_{k=0}^{\infty}\frac{1}{k+1}c_kx^{k+1}$



  \section{Kurven}
  \paragraph{Diff'bare Kurven}
  \begin{itemize}
  \item k regulär in $t \Leftrightarrow k'(t) \neq 0$, sonst singulär
  \item k singulär in t $\Rightarrow \nexists$ Tangentialvektor
  \item $k'(t)=(k'_1(t),...,)$ Tangentialvektor
  \item $T_k(t)=\frac{1}{\lVert k'(t)\rVert_2}$ Tangentialeinheitsvektor in t
  \end{itemize}


  \paragraph{Rektifizierbarkeit, Bogenlänge}\hspace{0pt} \\
  $k:[a,b]\rightarrow \real^n$ rektifizierbar $\Leftrightarrow \{\sum_{k=1}^{N}\lVert\gamma(t_k)-\gamma(t_{k+1})\rVert_2: a = t_0 < t_1 < ... < t_N = b$ Unterteilung v. $[a,b]\}$


  \paragraph{Bogenlänge stetig diff'barer Kurven}\hspace{0pt} \\
  $k: [a,b]\rightarrow \real^n$ stückweise stetig diff'bar $\Rightarrow L(k)=\int_{a}^{b}\lVert k'(t)\rVert_2 dt$

  \paragraph{Parametertransformation}\hspace{0pt} \\
  $ f: J \rightarrow I \ \text{Parametertransformation} \Leftrightarrow \ \text{bijektiv und stetig;}\\  f, f^{-1} k\text{-mal stetig diff'bar} \Rightarrow \mathscr{C}^k \ \text{Parametertransformation}$\\
  $\mathscr{C}^1 \text{-Par.transf.} f \ \text{orientierungstreu wenn} \ f'(t) > 0 \ \forall t; \\ \text{orientierungsumkehrend für} \ <$\\
  $ k \ \text{und}\ \tilde{k} \ \text{äquivalente Kurven wenn mit Par.transf.}\ f \ \text{gilt}\ \tilde{k} = k \circ f$ äquivalente Kurven $\rightarrow$ gleiche Bogenlänge\\


  \paragraph{Bogenlänge/Umparametrisierung}\hspace{0pt} \\
\begin{itemize}
	\item   $k: [a,b] \rightarrow \real^n \ \text{stetig diff'bare Kurve};\\ f:[c,d] \rightarrow [a,b] \ \mathscr{C}^1\text{-Par.transf.}; k \ \text{und}\ k \circ f \ \text{gleiche Länge} $
	\item $k$ regulär $\Rightarrow\;\exists$ orientierungserhaltende $\mathscr{C}^1$-Parametertransformation $f : [a,L(k)] \rightarrow [a,b]$, sodass diese Kurve $k: J \rightarrow \real, k := k \circ f$ "'mit Einheitsgeschwindigkeit läuft"', also $\lVert k'(t) \rVert = 1 \; \forall t \in [0,L(k)]$
	\item Man erhält dieses $f$ als Umkehrfunktion von $s \rightarrow \int_{a}^{s} \lVert k'(t)\rVert dt$
\end{itemize}

  \paragraph{Krümmung}
  \begin{itemize}
  	\item $k: I \rightarrow \real^2$ regulär in $t \in I \Rightarrow \kappa(t)$ \textbf{Krümmung} im Punkt $t$
  	\item $k: I \rightarrow \real^2$ regulär, zweimal stetig diff'bar, k die dazugehörige unparametrisierte Kurve $\Rightarrow$ Krümmung von k in t als $\tilde{k}$ an $\tilde{t}:\;k(t)=\tilde{k}(\tilde{t})$
  \end{itemize}
	$k: I \rightarrow \real^2$ zweimal stetig diff'bar, $t \in I: k$ in $t$ regulär
	\begin{itemize}
		\item k nach Bogenlänge parametrisiert $\Rightarrow \kappa(t)=\langle k''(t), N(t) \wedge \lvert \kappa(t)\rvert = \lVert T'(t) \rVert = \lVert k''(t) \rVert$
		\item $k(t) = 
		\begin{pmatrix}
			x(t)\\y(t)\\
		\end{pmatrix}
		\Rightarrow \kappa(t)=\frac{x'(t)y''(t)-y'(t)x''(t)}{\sqrt{(x'(t)^2+y'(t)^2)^3}}$
	\end{itemize}
  \section{Mehrdimensionale Differentialrechnung}
	$M \subseteq \real^n$ offen, $f: M \rightarrow \real, x \in M$ 
	\paragraph{Richtungsableitung}
	$ v \in \real^n \backslash \{0\}$ Richtungsvektor $\Rightarrow$ Richtungsableitung von $f$ in Richtung $v = \partial_vf(x)= \lim\limits_{x \rightarrow 0} \frac{f(x+tv)-f(x)}{t}$
	
  \paragraph{Totale Differenzierbarkeit}
$f$ in $x$ (total) diff'bar $\Leftrightarrow \exists$ lineare Abbildung $L: \real^n \rightarrow \real^n, h \in \real^n \backslash \{0\}: \lim\limits_{h \rightarrow 0} \frac{f(x+h)-f(x)-Lh}{\lVert h \rVert} = \lim\limits_{h \rightarrow 0} \frac{f(x+h)-f(x)-\langle w,h \rangle}{\lVert h \rVert} = 0$

 \paragraph{Diff'barkeit im Mehrdimensionalen}
 $f$ (total) diff'bar $\Leftrightarrow \forall x \in M$ alle partiellen Ableitungen existieren und stetig sind $\Rightarrow f$ stetig diff'bar, $w=\begin{pmatrix}
 \partial_1f(x)\\\vdots \\ \partial_nf(x)
 \end{pmatrix} =: \nabla f(x),\; \partial_vf(x)=\langle \nabla f(x), v \rangle$

 \paragraph{Hesse-Matrix}\hspace{0pt}\\
  $ \nabla^2f(x) = \begin{pmatrix}
  \partial_{11}f(x)  & \cdots & \partial_{1,n}f(x) \\
  \vdots  & \ddots & \vdots  \\
  \partial_{n1}f(x) & \cdots & \partial_{nn}f(x)
 \end{pmatrix}$



 \paragraph{Taylor in höheren Dimensionen}

  % TODO: maybe remove that many pmatrices
  
 $\arraycolsep=1.4pt\def\arraystretch{1}
 T_2f((x,y);(0,0))=f(0,0)+\langle f(0,0),
 \begin{pmatrix}
   x\\
   y
 \end{pmatrix}
 \rangle + \frac{1}{2}
 \begin{pmatrix}
   x&y
 \end{pmatrix}
 (\nabla^2f
 \begin{pmatrix}
   x&y
 \end{pmatrix}
 )
 \begin{pmatrix}
   x&y
 \end{pmatrix}
 $

  \paragraph{Extrema/Kritische Punkte}
  \begin{itemize}
  	\item kritischer Punkt: $\nabla f(c) = 0$
  	\item lokales Min.: $\nabla^2f(c)$ pos. semidefinit
  	\item lokales Max.: $\nabla^2f(c)$ neg. semidefinit
  	\item $\nabla f(c)=0 \ \text{und}\ \nabla^2f(c) \ \text{pos. definit}\ \Rightarrow$ isoliertes lok. Min.
  	\item  $\nabla f(c)=0 \ \text{und}\ \nabla^2f(c) \ \text{neg. definit}\ \Rightarrow$ isoliertes lok. Max.
  	\item $\nabla f(c)=0 \ \text{und}\ \nabla^2f(c) \ \text{pos. und neg. EW}\ \Rightarrow$ Sattelpunkt
  \end{itemize}

  \paragraph{Jacobi-Matrix}

  $F: \real^n \rightarrow \real^m; F(x) =
  \begin{pmatrix}
    f_1(x)\\
    \vdots \\
    f_m(x)
  \end{pmatrix}
  $ \\
  $f \ \text{in}\ c \ \text{(total) diff'bar wenn lin. Abb. (oder Matrix)}\ L \ \text{existiert mit:}\\ \displaystyle\lim_{\substack{h \rightarrow 0\\ h \in \real^n \setminus \{0\} }} \frac{F(x+h)-F(x)-Lh}{\|h\|} = 0$\\
  Alle $\partial_jf_k(x)$ ($j=1 \dots n,k=1 \dots m)$ in $c$ stetig $\Rightarrow$ $f$ in $c$ stetig diff'bar und Jacobi-Matrix \\
  $DF(c) :=
  \begin{pmatrix}
    \partial_1f_1(c) & \dots   & \partial_nf_1(c) \\
    \vdots           & \ddots  & \vdots          \\
    \partial_1f_m(c) & \dots   & \partial_nf_m(c) \\
  \end{pmatrix}
  $

  \paragraph{Mehrdimensionale Kettenregel}\hspace{0pt} \\
  $D(G \circ F)(c) = DG(F(c)) \cdot DF(c)$
  
  \section{Mehrdimensionale Integralrechnung}
  % Sazu von Fubini?

  \paragraph{Transformationssatz}
  $ T: M_2 \rightarrow M_1; f: M_1 \rightarrow \real $\\
  $ \idotsint \limits_{M_1}f(x)\mathrm{d}x = \idotsint \limits_{M_2}f(T(y)) \ |\mathrm{det}(DT(y))| \ \mathrm{d}y $


  \section{Differentialgleichungen}
  \paragraph{Rezepte}\hspace{0pt}\\

  \begin{tabular}{|l|l|l|l|}
    \hline
    1   & $y'(t)=f(t) \cdot g(y(t))$      & Funkt. $f,g$              \\ \hline
    2   & $y'(t)+a(t) \cdot y(t) = 0$     & Funkt. $a$                \\ \hline
    3   & $y'(t)+a(t) \cdot y(t) = f(t)$  & Funkt. $a,f$              \\ \hline
    4   & $y''(t)+ay'(t)+by(t) = 0$       & Konst. $a,b$              \\ \hline
    5   & $y''(t)+ay'(t)+by(t) = p(t)$    & Konst. $a,b;$ Polyn. $p$  \\ \hline
    6   & $y''(t)+ay'(t)+by(t) = e^{ct}$   & Konst. $a,b,c$            \\ \hline
  \end{tabular}

  \paragraph{Meth. 1}
  $F(t)=\int f(t) \mathrm{d}t;G(t)=\int \frac{1}{g(t)\mathrm{d}t}$
  Jede allgemeine Lsg für $y(t)$ erfüllt die Gleichung $G(y(t))=F(t)+c$
  Mit Anfangsbedingung erfüllt jede Lsg. $\int_{y_0}^{y(t)}\frac{1}{g(u)}\mathrm{d}u=\int_{x_0}^{t}f(u)\mathrm{d}u$

  \paragraph{Meth. 2}
  $A(t)=\int a(t)\mathrm{d}t$; allgemeine Lsg. $y(t)=ce^{-A(t)}$

  \paragraph{Meth. 3}
  $A(t)=\int a(t)\mathrm{d}t; B(t)=\int e^{A(t)} \cdot f(t) \mathrm{d}t$
  Allg. Lsg. $y(t)=e^{-A(t)} \cdot (c+B(t))$

  \paragraph{Meth. 4}
  falls: $a^2>4b$ Löse $\lambda_{1,2}=\lambda^2+a\lambda + b = 0$
  allg. Lsg.: $y(t)=c_1e^{\lambda_1t}+c_2e^{\lambda_2t}$ \\
  falls: $a^2=4b$ $\lambda = - \frac{a}{2}$
  allg. Lsg.: $y(t)=(c_1+c_2)e^{\lambda t}$\\
  falls: $a^2<4b$ $\omega = \sqrt{b - (\frac{a}{2})^2}$
  allg. Lsg.: $y(t)=(c_1cos(\omega t) + c_2sin(\omega t))e^{-\frac{a}{2}t}$

  \paragraph{Meth. 5}\hspace{0pt}\\
  Bestimme allg. Lsg. $y_h(t)$ von $y''_h(t)+ay'_h(t)+by_h(t)=0$\\
  Stelle Polyn. $q(t)=a_nt^n+\dots+a_1t+a_0$ mit Grad $n=deg(p) \ \text{und Param.}\ a_0,\dots,a_n$ auf \\
  $y_p(t) =
  \begin{cases}
    q(t)    & \quad b \neq 0 \\
    tq(t)   & \quad a \neq 0, b = 0\\
    t^2q(t) & \quad a = 0, b = 0\\
  \end{cases}
  $ \\
  in Abhängigkeit von $a_0,\dots ,a_n$;
  $y''_p(t)+ay'_p(t)+by_p(t)=p(t)$
  ermittle $a_1,\dots,a_n$ durch Koeffizientenvergleich \\
  Allg. Lsg.: $y(t)=y_h(t)+y_p(t)$

  %TODO: compare with slides from Carlos at a later point in time. is this right? 

    \paragraph{Meth. 6}\hspace{0pt}\\
  Bestimme allg. Lsg. $y_h(t)$ von $y''_h(t)+ay'_h(t)+by_h(t)=0$\\
  Löse $\lambda^2+a\lambda+b=0$
  $y_p(t) =
  \begin{cases}
    ke^{ct}    & \quad \lambda_1 \neq c \neq \lambda_2 \\
    kte^{ct}   & \quad \lambda_1 = c \neq \lambda_2 \ \text{oder}\ \lambda_1 \neq c = \lambda_2 \\
    kt^2e^{ct} & \quad  \lambda_1 = c = \lambda_2 \\
  \end{cases}
  $ \\
  in Abhängigkeit von $k$;
  $y''_p(t)+ay'_p(t)+by_p(t)=e^{ct}$
  ermittle $k$\\
  Allg. Lsg.: $y(t)=y_h(t)+y_p(t)$
  % TODO: compare with slides from Carlos at a later point in time. is this right?
  
  \paragraph{Mehrdimensionale DGLs}
  Sei $\{v_1, \dots, v_n\} \subset \real^n$ Basis von Eigenvektoren von $A \in \real^{n\times n}$ zu EW $\lambda_1, \dots, \lambda_n$\\
  Lin. unabh. Lösungen v. DGL der Form $y'=Ay$: $y=e^{\lambda_jt}v_j$
  
  \section{Komplexe Zahlen}
  $\compl = \{z \vert z=x+iy, x,y \in \real, i^2=-1\}$
  \begin{itemize}
  	\item $z_1+z_2 = (x_1 \pm x_2)+i(y_1 \pm y_2)$
  	\item $z_1 \cdot z_2 = x_1x_2-y_1y_2+i(x_1y_2+x_2y_1)$
  	\item $\displaystyle \frac{z_1}{z_2} = \frac{x_1x_2+y_1y_2}{x_2^2+y_2^2} + i \frac{x_2y_1-x_1y_2}{x_2^2+y_2^2} = \frac{z_1\overline{z_2}}{\lvert z_2 \rvert^2}$
  	\item $\lvert z \rvert = \sqrt{x^2+y^2}$
  	\item $\overline{z} = x-iy$
  \end{itemize}
\end{multicols*}
\end{document}

% Further TODOs:
% - Logarithmus Reihe
% - Bekannte Integrale und Ableitungen

%%% Local Variables:
%%% mode: latex
%%% TeX-master: t
%%% TeX-engine: xetex
%%% End:
